\documentclass[11pt]{article}


\usepackage{url}
\usepackage[hyperfootnotes=false]{hyperref}
\usepackage[font=footnotesize,labelfont=bf]{caption}
\usepackage[usenames, dvipsnames]{color}
\definecolor{navyblue}{rgb}{0.0, 0.0, 0.5}
\usepackage{graphicx}
\usepackage[scientific-notation=true]{siunitx}
\usepackage{subcaption}
\usepackage{setspace}
\usepackage[top=0.6 in, bottom=0.7 in, left=0.7in, right=0.7in]{geometry}
\usepackage{setspace}
\usepackage{vector}
\usepackage{amsmath}
\usepackage{amssymb}
\usepackage{tgtermes}
\DeclareGraphicsExtensions{.pdf,.png,.jpg}

\usepackage{epstopdf}

\title{\textbf{Systematics from Correction Factors}}

\begin{document}
\maketitle

Starting from the experimental cross section,

\begin{equation}
  \sigma^{exp} = \sigma^{exp}_{uncorr} \cdot f_{1} \cdot f_{2} . . . \cdot f_{n}
\end{equation}
where $\sigma^{exp}_{uncorr}$ is the uncorrected Yield divided by the SIMC phase space, and $f_{n}'s$ are the correction factors,

\begin{align*}
  f_{1} &= \frac{1}{1-m\cdot I_{avg}}   \hspace{2cm}  \text{        target boiling factor, where $m$ is the slope} \\
  f_{2} &= \frac{1}{p_{T}}              \hspace{4cm}   \text{proton transmission factor}  \\
  f_{3} &= \frac{1}{\epsilon_{eTrk}}   \hspace{4cm}   \text{e-  tracking efficiency} \\
  f_{4} &= \frac{1}{\epsilon_{hTrk}}   \hspace{4cm}   \text{h  tracking efficiency} \\
  f_{5} &= \frac{1}{\epsilon_{tLT}}   \hspace{4cm}   \text{total live time} \\
  f_{6} &= \frac{1}{Q_{tot}}          \hspace{4cm}   \text{total charge} \\
  f_{7} &= f_{rad} \hspace{4cm}   \text{radiative correction factor} \\
  f_{8} &= f_{bc}  \hspace{4cm}   \text{bin-centering correction factor} \\
\end{align*}
The systematic uncertainty on the cross section due to the uncertainty in each of these correction factors is
\begin{align}
  (d\sigma^{exp}_{syst})^{2} = \sum_{i=1}^{8} \Big( \frac{\partial\sigma^{exp}}{\partial f_{i}}\Big)^{2} df_{i}^{2}
\end{align}
If the derivative with respect to factor $f_{i}$ is
\begin{equation}
  \frac{\partial\sigma^{exp}}{\partial f_{i}} = \frac{\sigma^{exp}}{f_{i}}
\end{equation}
Substituting (3) in (2), one obtains
\begin{align}
  (d\sigma^{exp}_{syst})^{2} = (\sigma^{exp})^{2}\sum_{i=1}^{8} \Big( \frac{df_{i}}{f_{i}}\Big)^{2}
\end{align}
For a given data set with multiple runs, the correction factor might differ from run to run, for example, the tracking efficiencies or total
live time might be slightly different. In this case, the correction factor and its uncertainty is calculated for each run.  The weighted average is
then determined as follows: \\
Define a weight, $w$, 
\begin{equation}
  w_{i} \equiv \frac{1}{df_{i}^{2}}
\end{equation}
where the sum is over all run numbers, and $\sigma_{i}$ is the uncertainty in the correction factor $f_{i}$ for the $i$ run.
The weighted average and its uncertainty on the correction factor can then be expressed as
\begin{align}
  f_{w} &= \frac{\sum_{i}f_{i}w_{i}}{\sum_{i}w_{i}} \\
  d_{fw} &= \frac{1}{\sqrt{\sum_{i}w_{i}}}
\end{align}
where (6) and (7) are the correction factor and its uncertainty for a given experimental data set.
Substituting (6) and (7) in (4) would then give the total systematic uncertainty for  that particular data set.
Each of the $(P_{m}, \theta_{nq})$ for that data set would have a specific cross section, $\sigma_{exp}$, but all
bins would have the same systematic error, $d_{fw}$.

\end{document}
