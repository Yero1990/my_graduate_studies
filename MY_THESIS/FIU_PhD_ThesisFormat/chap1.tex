\chapter{INTRODUCTION}

\section{Motivation}

{\em Finite automata} are among the simplest mathematical models
of computation, and since the 1940s their theory has been one of the
cornerstones of theoretical computer science.
The seminal 1959 paper by Rabin and Scott \cite{rabinscott59}
introduced {\em nondeterministic finite automata},
in which the state transitions are not necessarily uniquely determined
by the current state and input symbol; instead the automaton is
imagined to have a magical ability to ``guess'' the correct transition
at each step.

\begin{figure}[t]
\fbox{HERE IS A TEXT FIGURE.}
\caption{The caption on this figure is so long that it cannot fit
onto a single line, instead requiring a line break.}
\end{figure}

\begin{figure}
\fbox{HERE IS ANOTHER TEXT FIGURE.}
\caption[Abbreviated caption]{The caption on this figure is also
extremely long, but in this case we have included an optional
argument that gives an abbreviated version of the caption, which
is what will be shown in the List of Figures.}
\end{figure}

As an illustration of \LaTeX's mathematics formatting,
here is the definition of {\em R\'enyi entropy}:
\begin{equation}
H_{\alpha}(X) =
\frac{1}{1-\alpha}
\log \left(\sum_{x \in {\cal X}}P[X=x]^{\alpha} \right) .
\end{equation}

