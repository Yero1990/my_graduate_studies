\documentclass[a4paper]{article}
%\usepackage{simplemargins}

%\usepackage[square]{natbib}
%\usepackage{amsmath}
%\usepackage{amsfonts}
%\usepackage{amssymb}
%\usepackage{graphicx}
%\usepackage[T1]{fontenc}
%\usepackage[utf8]{inputenc}
\usepackage[english]{babel}
%\usepackage[urw-garamond]{mathdesign}
\usepackage[T1]{fontenc}
\usepackage[utf8]{inputenc}

\usepackage{tgbonum}
\usepackage[left=4cm, right=4cm, top=1cm]{geometry}

\begin{document}
\pagenumbering{gobble}

\Large
 \begin{center}
\textbf{First Cross Section Results of D(e,e'p)n at Very High Recoil Momenta\\} 

\hspace{10pt}

% Author names and affiliations
\large
Carlos Yero \\

\hspace{10pt}

\small
Spokespeople: Drs. Werner U. Boeglin and Mark K. Jones

\hspace{10pt}

\small
(Dated: July 01, 2019) \\

\hspace{10pt}

\small  
Florida International University\\

\end{center}

\hspace{10pt}

\large
\begin{center}
ABSTRACT
\end{center}
\normalsize

\noindent Preliminary D(e,e'p)n electro-disintegration cross sections at $Q^{2} = 4.25$ GeV$^{2}$ with recoil momenta up to 900 MeV/c will be
presented. The experiment ran for a total of 6 days of beam time in April 2018 at Jefferson Lab in Hall C
and it seeks to study the short range structure of the deuteron by probing its high momentum tails
beyond 500 MeV/c, where currently no data exists. The experiment was part of a group of Hall C experiemnts that commissioned the
the new Hall C Super High Momentum Spectrometer (SHMS). At these kinematics, Meson Exchange Currents (MEC) and Isobar Configurations (IC)
are suppressed. Final State Interactions (FSI), however, can be important. From theoretical calculations by M. Sargsian in
"Modern Studies of the Deuteron: from the Lab Frame to the Light Front" (Boeglin, W.U. and Sargsian, M.M. Int.J.Mod.Phys. 2015),
FSI can be suppressed by choosing a kinematic region where the neutron recoil angle is between 35 and 45 degrees. This suppression
was seen in a previous D(e,e'p)n experiment that went to recoil momentum of 500 MeV/c. See for example, "Probing the High Momentum Component
of the Deuteron at High Q2" (Boeglin, W.U. et al. Phys Rev Lett. 2011). In this region, the Plane Wave Impulse Approximation (PWIA) is valid
and comparisions between measured and predicted cross sections have a sensitivity to the different NN potentials. Comparisons between the data
and different theory calculations with different NN potentials will be shown.



\end{document}
