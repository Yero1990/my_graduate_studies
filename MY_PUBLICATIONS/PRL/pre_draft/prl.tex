%% ****** Start of file template.aps ****** %
%%
%%
%%   This file is part of the APS files in the REVTeX 4 distribution.
%%   Version 4.0 of REVTeX, August 2001
%%
%%
%%   Copyright (c) 2001 The American Physical Society.
%%
%%   See the REVTeX 4 README file for restrictions and more information.
%%
%
% This is a template for producing manuscripts for use with REVTEX 4.0
% Copy this file to another name and then work on that file.
% That way, you always have this original template file to use.
%
% Group addresses by affiliation; use superscriptaddress for long
% author lists, or if there are many overlapping affiliations.
% For Phys. Rev. appearance, change preprint to twocolumn.
% Choose pra, prb, prc, prd, pre, prl, prstab, or rmp for journal
%  Add 'draft' option to mark overfull boxes with black boxes
%  Add 'showpacs' option to make PACS codes appear
\documentclass[aps,prl,twocolumn,showpacs,superscriptaddress,groupedaddress]{revtex4-2}  % for review and submission
%\documentclass[aps,preprint,showpacs,superscriptaddress,groupedaddress]{revtex4}  % for double-spaced preprint
\usepackage{graphicx}  % needed for figures
\usepackage{dcolumn}   % needed for some tables
\usepackage{bm}        % for math
\usepackage{amssymb}   % for math
\usepackage[hidelinks]{hyperref}
\usepackage{fancyhdr}
\usepackage{xcolor}
\usepackage{float}
\usepackage{lipsum}
\hypersetup{
    colorlinks,
    linkcolor={blue!80!black},
    citecolor={blue!80!black},
    urlcolor={blue!80!black}
}
% avoids incorrect hyphenation, added Nov/08 by SSR
\hyphenation{ALPGEN}
\hyphenation{EVTGEN}
\hyphenation{PYTHIA}


\begin{document}

% The following information is for internal review, please remove them for submission
\widetext
\leftline{Version xx as of \today}
\leftline{Primary authors: Carlos Yero, Werner Boeglin, Mark Jones}
\leftline{To be submitted to PRL}
\leftline{Comment to {\tt cyero002@fiu.edu} by xxx, yyy}
\centerline{\em D\O\ INTERNAL DOCUMENT -- NOT FOR PUBLIC DISTRIBUTION}

% the following line is for submission, including submission to the arXiv!!
%\hspace{5.2in} \mbox{Fermilab-Pub-04/xxx-E}

\title{First Measurements of the D(e,e'p)n Cross Section at \\ Very High Recoil Momenta and Large Q^{2}}
\input author_list.tex       % D0 authors (remove the first 3 lines
                             % of this file prior to submission, they
                             % contain a time stamp for the authorlist)
                             % (includes institutions and visitors)
\date{\today}


\begin{abstract}
  First results of cross section measurements of the $^{2}H(e,e'p)n$ reaction at 4-momentum transfers \\ 4 $\leq Q^{2}\leq$ 5 (GeV/c)$^{2}$  and
  neutron recoil momenta up to 1.18 GeV/c are presented. At the selected kinematics, Meson Exchange Currents (MEC) and Isobar
  Configurations (IC) are suppressed. Final State Interactions (FSI) have also been suppressed by choosing a kinematic region where the neutron recoil
  angle ($\theta_{nq}$) is between 35 and 45 degrees with respect to the 3-momentum transfer, $\vec{q}$. In this region, the Plane Wave Impulse Approximation (PWIA)
  dominates and comparison to recent theoretical calculations show data to be sensitive to momentum distributions up to $\sim$ 700 MeV/c recoil momenta.
  
  
%  momentum distribution has been determined from measured for neutron recoil momenta up to 1.14 GeV/c.
  
  %An article usually includes an abstract, a concise summary of the work
%covered at length in the main body of the article. It is used for
%secondary publications and for information retrieval purposes.
%For PRL, the rule of thumb is that the abstract should be less than
%8 lines and the text (excluding authors, abstract but including tables,
%figures and references) should be less than 4 pages (leave about 20 lines
%empty on page 4) in two-column format.
%PRL and PRD papers have to have PACS (Phsyics and Astronomy Classification
%Scheme) numbers. Please see {\tt http://www.aip.org/pacs/} for the numbers
%relevant to your paper. A set of standard references can be found at the
%end of this example paper.
\end{abstract}

\pacs{}
\maketitle
%\thispagestyle{plain} 

%\section{\label{sec:level1}First-level heading}
% sections are not used for PRL papers

%--------------MOTIVATION / introduction (from my proposal)--------------

\input introduction.tex
\input body.tex
\input acknowledgement.tex    % input acknowledgement
\bibliography{prl}
.\\\\\\\\\\\\\\\\\\\\\\\\\\\\\\\\\\\\\\\\\\\\\\\\\\\\\\\\\\\\\\\\\\\\\\\\\\\\\\\\\\\\\\\\\\\\\\\\\\\\\\\\\\\\\\\\\\\\\\\\\\\\\\\\\\\\\\\\\\\\\\\\\\\\\\\\\\\\\
\end{document}
%
% ****** End of file template.aps ******
