The deuteron is the only two-nucleon bound system. Therefore, it serves as a starting point to study the strong nuclear force at the subfermi distance scale, a region which is currently
not well understood. At such small internucleon distances the NN (nucleon-nucleon) potential is expected to exhibit a repulsive core in which the interacting
nucleon pair begins to overlap. The overlap is directly related to two-nucleon short range correlations (SRC) observed in $A>2$ nuclei \cite{PhysRevC.68.014313,PhysRevLett.96.082501,PhysRevLett.99.072501,Fomin_2017}.
Short-range studies of the deuteron are also important in determining whether or to what extent the description of nuclei in terms of nucleon/meson degrees of freedom is still valid before
having to include explicit quark degree of freedoms, an issue of fundamental importance in nuclear physics\cite{pr01-020}. As of the present time, there are only a few nuclear physics experiments for
which a transition between nucleonic to quark degrees of freedom been observed \cite{PhysRevLett.81.4576,PhysRevLett.87.102302,PhysRevC.66.042201}.\\
\indent The most direct way to study the short range structure of the deuteron wavefunction (or equivalently, its high momentum components) is via the exclusive deuteron
electro-disintegration reaction at very high neutron recoil (or missing) momenta. Within the PWIA, the virtual photon couples to
the proton which is subsequently ejected from the nucleus without further interaction with the recoiling neutron. The neutron carries a momentum equal in magnitude but opposite in direction
to the initial state proton, $\vec{p}_{r} = -\vec{p}_{i,p}$, thus providing information on the momentum of the bound nucleon and its momentum distribution. \\
\indent In reality, the ejected particles undergo subsequent interactions resulting in re-scattering between the proton and neutron (FSIs). Another possibility is that the
photon may couple to the virtual meson being exchanged between the nucleons (MECs), or the photon may excite either nucleon in the deuteron into a resonance state (ICs) which
decays back into the ground state nucleon causing further re-scattering between the proton and neutron. Both MECs and ICs in addition to FSIs can significantly alter the recoiling neutron
momentum thereby obscuring any possibility of directly accessing the deuteron momentum distributions. \\
\indent Theoretically, MECs and ICs are expected to be suppressed at $Q^{2}>1$ (GeV/c)$^{2})$ and Bjorken $x_{Bj}\equiv Q^{2}/2M_{p}\omega>1$, where $M_{p}$ and $\omega$ are the proton mass and photon energy transfer, respectively.
The suppression of MECs can be understood from the fact that the estimated MEC scattering amplitude is proportional to  $(1 + Q^{2}/m^{2}_{meson})^{-2}(1+Q^{2}/\Lambda^{2})^{-2}$, where $m_{meson}\approx0.71$ (GeV/c)$^{2}$ and
$\Lambda^{2}\sim 0.8-1 $ (GeV/c)$^{2}$\cite{Sargsian_2001}. The ICs can be suppressed kinematically by selecting $x_{Bj}>1$, where one probes the lower part of the deuteron quasi-elastic peak which is maximally away from the inelastic resonance
electro-production threshold. \\
\indent For FSIs at large $Q^{2}$, the onset of the General Eikonal Approximation (GEA)\cite{Sargsian_2001,PhysRevC.56.1124,sargsian_2015} is expected which predicts a strong angular dependence of the FSIs with neutron recoil angles where FSI peaks at $\theta_{nq}\sim70^{o}$. The
most important prediction from GEA, however, is that at large recoil momenta $p_{r}$ where FSIs are expected to be large, there is an approximate cancellation of the PWIA/FSI interference (screening term) with the
modulus-squared of the FSI amplitude (rescattering term). This cancellation results in only the PWIA term remaining in the deuteron cross section and is expected to occur at netron recoil angles $\theta_{nq}\sim40^{o}$ and $\theta_{nq}\sim120^{o}$. Since at $\theta_{nq}\sim120^{o}$ ICs are not negligible, $x_{Bj}>1$ is required to suppress ICs which leaves $\theta_{nq}\sim40^{o}$ as the only choice where FSIs are reduced. \\
\indent Previous deuteron electro-disintegration experiments performed at Jefferson Lab (JLab) have helped confirmed various of the abovementioned theoretical predictions as well as constrain and quantify the contributions from FSIs, MECs and ICs on
the $^{2}H(e,e'p)n$ cross-section to determine the kinematics at which they are either suppressed (MECs and ICs) or under control (FSIs). The first of these was performed in Hall A \cite{PhysRevLett.89.062301}
at a relatively low momentum transfer of $Q^{2}=0.665$ (GeV/c)$^{2}$ and neutron recoil momenta up to $p_{r} = $ 550 MeV/c where it was shown that for $p_{r}>$ 300 MeV/c, FSIs, MECs and ICs dominate the cross section and
had to be included in Arenh\"{o}vel's calculations \cite{PhysRevC.43.1022, PhysRevC.46.455, PhysRevC.52.1232, PhysRevC.55.2214} for a satisfactory agreement between theory and data. \\
\indent The next experiment was performed in Hall B \cite{PhysRevLett.98.262502} using the CEBAF Large Acceptance Spectrometer (CLAS) which measured a wide variety of kinematic settings
giving an overview of the $^{2}H(e,e'p)n$ reaction kinematics. This was the first experiment to probe
the deuteron at high momentum transfers ( 1.75 $\leq Q^{2}\leq$ 5.5 (GeV/c)$^{2}$) and presented angular distributions of cross-sections that exhibited a strong angular dependence of FSI with neutron recoil angles peaking
at $\theta_{nq} \sim 70^{o}$ which confirmed the onset of the GEA\cite{Sargsian_2001,PhysRevC.56.1124}.
The cross sections versus neutron recoil momenta up to $p_{r}\sim$2 GeV/c were also presented, however, statistical limitations made it necessary to integrate over a wide angular range making it impossible to control
contributions from FSIs, MECs and ICs. \\
\indent Finally, a third $^{2}H(e,e'p)n$ experiment was performed in Hall A \cite{PhysRevLett.107.262501} at $Q^{2} = 3.5\pm0.25$ (GeV/c)$^{2}$ and recoil momenta up to 550 MeV/c. The angular distributions of
the cross-section ratio ($R = \sigma_{exp}/\sigma_{PWIA}$) presented saw agreement in shape with the angular distributions in Hall B\cite{PhysRevLett.98.262502} with FSI peaking at $\theta_{nq}\sim70^{o}$. With the increased statistics in Hall A, a direct comparison of the momentum distributions between data and theory could be made for specific $\theta_{nq}$ bins. For the first time, at $\theta_{nq}=35\pm5^{o}$ and $45\pm5^{o}$, theoretical models showed a sensitivity to the NN-potential used in the deuteron wavefunction that was not obscured by FSI contributions to the cross section data.\\
\indent The experiment presented in this Letter takes advantage of the kinematic window previously found in Hall A\cite{PhysRevLett.107.262501} and extends the $^{2}H(e,e'p)n$ cross section measurements
to $Q^{2}=4.5\pm0.5$ (GeV/c)$^{2}$ and neutron recoil momenta up to 1.18 GeV/c. At these kinematics, MECs and ICs are suppressed and FSIs are under control for neutron recoil angles between 35 and 45 degrees
giving access to unprecedented high momentum components of the deuteron wavefunction. \\
