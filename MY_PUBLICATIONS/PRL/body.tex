\indent This experiment was part of a group of four experiments that commissioned the new Hall C Super High Momentum Spectrometer (SHMS) as part of the 12 GeV upgrade at JLab.
An electron beam was incident on a 10 cm long liquid deuterium target (LD2). The scattered electron and knocked-out proton were detected in coincidence
by the SHMS ans High Momentum Spectrometer (HMS), respectivly. The ``missing'' (undetected) neutron was reconstructed from momentum conservation laws.
The beam currents delivered by the accelerator ranged between 40-60 $\mu$A due to frequent beam trips at higher currents and the beam was rastered over a 2x2 mm$^{2}$ area to reduce
the effects of localized boiling on the cryogenic targets (hydrogen and deuterium). \\
\indent Both spectrometers at Hall C have similar standard detector packages, each with 1) four sets of hodoscope planes (scintillator arrays) used for triggering,
2) a pair of drift chambers used for tracking, 3) a calorimeter used for $e^{-}/\pi^{-}$ discrimination and 4) a gas \u{C}erenkov used also for $e^{-}/\pi^{-}$ and an additional
Noble Gas \u{C}erenkov used in the SHMS for $e^{-}/\pi^{+}/K^{+}$ at momenta $>$ 6 GeV/c. Due to the absence of significant background on this experiment and the low coincidence trigger rates
($\sim 1-3$ Hz) at the higher missing momentum settings, the use of additional particle identification (PID) was found to have little to no effect on the final cross section. \\
\indent The kinematics measured in our experiment consisted of three missing momentum settings: $p_{r}=80,580$ and $750$ MeV/c.
The spectrometer settings were as follows: the SHMS central angle and momentum settings were kept fixed at (12.194 deg, 8.5342 GeV/c) and the HMS central angle and momentum settings were changed from
(38.896 deg, 2.840 GeV/c) at the 80 MeV setting to (54.992 deg, 2.1925 GeV/c) and (58.391 deg, 2.0915 GeV/c) at the 580 and 750 MeV/c settings, respectively. At these kinematics, the
3-momentum transfer is $|\vec{q}| = 2.86$ GeV/c and is on the order of the final proton momentum indicating that most of the energy and momentum are transferred to the proton. As a result, the ejected proton
scatters at angles $\theta_{pq}\sim 0$, relative to the $\vec{q}$. This configuration is known as the ``parallel-kinematics'' and suppresses the process in which the neutron is struck and the proton is a spectator. \\
\indent In addition to deuteron kinematics,  $^{1}H(e,e'p)$ data was also taken at kinematics close to the deuteron 80 MeV setting for cross-checks with the spectrometer acceptance model as well as for normalization purposes using the  Hall C Monte Carlo
simulation program, \texttt{SIMC}. Additional $^{1}H(e,e'p)$ data was also taken at three other kinematic settings to cover the entire SHMS momentum acceptance range for spectrometer optics studies and
calibration. \\
\indent The event selection criteria was determined by making 1) standard cuts on the spectrometer momentum fraction ($\delta$) to select a region in which the reconstruction optics
is well known, 2) an HMS collimator cut to restrict the spectrometer solid angle acceptance to events that only passed through the collimator and not by re-scattering from the edges, 3) a missing
energy cut (peak $\sim$ 2.2 MeV for the deuteron) to select true $ep$ coincidences and not events from the radiative tail, 4) a coincidence time cut to select true coincidence events and not accidentals,  5) a PID cut on the
SHMS calorimeter to select electrons and not other sources of background, mostly pions and 6) a z-vertex difference cut between the HMS and SHMS $z$ reaction vertex difference to select events that truly
originated from the same reaction vertex at the target.
\indent The data yield was corrected for tracking efficiencies, total live time, proton absorption and target boiling factors.


