\indent This experiment was part of a group of four experiments that commissioned the new Hall C Super High Momentum Spectrometer (SHMS) as part of the 12 GeV upgrade at JLab.
An electron beam was incident on a 10 cm long liquid deuterium target (LD2). The scattered electron and knocked-out proton were detected in coincidence
by the SHMS ans High Momentum Spectrometer (HMS), respectivly. The ``missing'' (undetected) neutron was reconstructed from momentum conservation laws.
The beam currents delivered by the accelerator ranged between 40-60 $\mu$A due to frequent beam trips at higher currents and the beam was rastered over a 2x2 mm$^{2}$ area to reduce
the effects of localized boiling on the cryogenic targets (hydrogen and deuterium). \\
\indent Both spectrometers at Hall C have similar standard detector packages, each with 1) four sets of hodoscope planes (scintillator arrays) used for particle triggering, 2) a pair of drift chambers used for
tracking, 3) a calorimeter used for $e^{-}/\pi^{-}$ discrimination and 4) a gas \u{C}herenkov used for additional $e^{-}/\pi^{-}$ and $\pi^{+}/K^{+}$ (only Heavy Gas \u{C}herenkov in SHMS) discrimination.
Given the low trigger rates and absence of significant background on this experiment, the use of additional particle identification (PID) cuts was found to have little to no effect on the final
cross section.
\indent 
