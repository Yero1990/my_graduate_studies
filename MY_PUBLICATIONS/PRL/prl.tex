%% ****** Start of file template.aps ****** %
%%
%%
%%   This file is part of the APS files in the REVTeX 4 distribution.
%%   Version 4.0 of REVTeX, August 2001
%%
%%
%%   Copyright (c) 2001 The American Physical Society.
%%
%%   See the REVTeX 4 README file for restrictions and more information.
%%
%
% This is a template for producing manuscripts for use with REVTEX 4.0
% Copy this file to another name and then work on that file.
% That way, you always have this original template file to use.
%
% Group addresses by affiliation; use superscriptaddress for long
% author lists, or if there are many overlapping affiliations.
% For Phys. Rev. appearance, change preprint to twocolumn.
% Choose pra, prb, prc, prd, pre, prl, prstab, or rmp for journal
%  Add 'draft' option to mark overfull boxes with black boxes
%  Add 'showpacs' option to make PACS codes appear
\documentclass[aps,prl,twocolumn,showpacs,superscriptaddress,groupedaddress]{revtex4-1}  % for review and submission
%\documentclass[aps,preprint,showpacs,superscriptaddress,groupedaddress]{revtex4}  % for double-spaced preprint
\usepackage{graphicx}  % needed for figures
\usepackage{dcolumn}   % needed for some tables
\usepackage{bm}        % for math
\usepackage{amssymb}   % for math
\usepackage[hidelinks]{hyperref}
\usepackage{xcolor}
\hypersetup{
    colorlinks,
    linkcolor={red!50!black},
    citecolor={blue!80!black},
    urlcolor={blue!80!black}
}
% avoids incorrect hyphenation, added Nov/08 by SSR
\hyphenation{ALPGEN}
\hyphenation{EVTGEN}
\hyphenation{PYTHIA}


\begin{document}

% The following information is for internal review, please remove them for submission
\widetext
\leftline{Version xx as of \today}
\leftline{Primary authors: Carlos Yero, Werner Boeglin, Mark Jones}
\leftline{To be submitted to PRL}
\leftline{Comment to {\tt cyero002@fiu.edu} by xxx, yyy}
\centerline{\em D\O\ INTERNAL DOCUMENT -- NOT FOR PUBLIC DISTRIBUTION}

% the following line is for submission, including submission to the arXiv!!
%\hspace{5.2in} \mbox{Fermilab-Pub-04/xxx-E}

\title{First Measurements of the D(e,e'p)n Cross Section at \\ Very High Recoil Momenta and Large Q^{2}}
\input author_list.tex       % D0 authors (remove the first 3 lines
                             % of this file prior to submission, they
                             % contain a time stamp for the authorlist)
                             % (includes institutions and visitors)
\date{\today}


\begin{abstract}
  First results of cross section measurements of the $^{2}H(e,e'p)n$ reaction at 4-momentum transfers \\ 4 $\leq Q^{2}\leq$ 5 (GeV/c)$^{2}$  and
  neutron recoil momenta up to 1.18 GeV/c are presented. At the selected kinematics, Meson Exchange Currents (MEC) and Isobar
  Configurations (IC) are suppressed. Final State Interactions (FSI) have also been suppressed by choosing a kinematic region where the neutron recoil
  angle ($\theta_{nq}$) is between 35 and 45 degrees with respect to the 3-momentum transfer, $\vec{q}$. In this region, the Plane Wave Impulse Approximation (PWIA) dominates and comparison
  to recent theoretical calculations show data to be sensitive to momentum distributions up to $\sim$ 700 MeV/c recoil momenta.
  
  
%  momentum distribution has been determined from measured for neutron recoil momenta up to 1.14 GeV/c.
  
  %An article usually includes an abstract, a concise summary of the work
%covered at length in the main body of the article. It is used for
%secondary publications and for information retrieval purposes.
%For PRL, the rule of thumb is that the abstract should be less than
%8 lines and the text (excluding authors, abstract but including tables,
%figures and references) should be less than 4 pages (leave about 20 lines
%empty on page 4) in two-column format.
%PRL and PRD papers have to have PACS (Phsyics and Astronomy Classification
%Scheme) numbers. Please see {\tt http://www.aip.org/pacs/} for the numbers
%relevant to your paper. A set of standard references can be found at the
%end of this example paper.
\end{abstract}

\pacs{}
\maketitle

%\section{\label{sec:level1}First-level heading}
% sections are not used for PRL papers

%--------------MOTIVATION / introduction (from my proposal)--------------

Being the most simple $np$ bound state, the deuteron serves as a starting point to study the
strong nuclear force at the subfermi level which is currently
not well understood. At such small internucleon distances 
the NN (nucleon-nucleon) potential is expected to exhibit a repulsive core in which 
the interacting nucleon pair begins to overlap. The overlap is 
directly related to two-nucleon short range correlations (SRC) observed in $A\geq2$
nuclei. Short-distance studies of the deuteron are
also important in determining whether or to what extent the
description of nuclei in terms of nucleon/meson degrees of 
freedom must be supplemented by the inclusion of explicit
quark effects, which is an issue of fundamental importance in
nuclear physics\cite{pr01-020}. \\
\indent The most direct way to study the short range structure (or equivalently, the high momentum components)
of the deuteron wavefunction is via the exclusive deuteron electro-disintegration reaction within the
PWIA kinematics. In this approximation, the virtual photon couples to the proton which is ejected from the nucleus without
further interaction with the recoiling neutron, which carries a momentum equal in magnitude but opposite in direction to the ejected proton, $\vec{p}_{r} = -\vec{p}_{i,p}$.
This gives direct access to the deuteron momentum distributions since the scattered neutron momentum remains unchanged. \\
\indent In reality, the ejected particles undergo subsequent interactions resulting in re-scattering of the proton and neutron (FSIs).
Another possibility is that the photon may couple to the virtual meson being exchanged between the nucleons (MECs), or the photon may excite either nucleon in the deuteron
into a resonance state (ICs) which decays back into the ground state nucleon causing futher re-scattering between the proton and neutron. The above-mentioned long-range processes
alter the final neutron momentum making the deuteron momentum distributions unaccessible.\\
\indent Previous deuteron electro-disintegration experiments performed at Jefferson Lab have helped dis-entangle and quantify the contributions from FSI, MEC and IC on the $^{2}H(e,e'p)n$ cross-section. 
The first of these was performed in Hall A at relatively low momentum transfers $Q^{2}=0.665$ (GeV/c)$^{2}$ and neutron recoil momenta up to $p_{r} = $ 550 MeV/c where it was shown that for
$p_{r}>$ 300 MeV/c, the inclusion of FSI, MEC and IC was necessary in Arenhovel's calculations for a satisfactory agreement between the theory and data. \\
\indent The next experiment was performed in Hall B using the CEBAF Large Acceptance Spectrometer (CLAS) which took advantage of its large detector acceptance to simultaneously measure
a wide variety of kinematic settings giving an overview of the $^{2}H(e,e'p)n$ reaction kinematics. This was the first experiment to probe the deuteron at high momentum transfers ( 1.75 $\leq Q^{2}\leq$ 5.5 (GeV/c)$^{2}$)
and presented angular distributions of cross-sections and confirmed the onset of the General Eikonal Approximation (GEA), which predicts a strong angular dependence of FSI with neutron recoil angles with FSI peaking at $\theta_{nq} \sim 70^{o}$.
The cross-sections versus neutron recoil momenta up to 2 GeV/c were also presented with integrated over all neutron recoil angles to gain better statistical precision. As a result, it was not possible to choose kinematical regions
regions in which FSI were minimal to extract the momentum distributions. \\
\indent Finally, a third $^{2}H(e,e'p)n$ experiment was performed in Hall A at $Q^{2} = 0.8, 2.5, 3.5$ (GeV/c)$^{2}$ and recoil momenta up to 550 MeV/c at kinematics which allowed the extraction of angular and momentum distributions for significantly
smaller kinematical bins than in Hall B/CLAS. The angular distributions were presented as the cross-section ratio, $R = \sigma_{exp}/\sigma_{PWIA}$ versus $\theta_{nq}$, and verified the strong anisotropy of FSI with $\theta_{nq}$.
Most importantly, for recoil neutron momentum bins, $p_{r}=0.4\pm0.02$ and $0.5\pm0.02$ GeV/c, the ratio $R\sim1$ for $35^{o}\leq \theta_{nq}\leq45^{o}$ indicating a reduced sensitivity of the experimental cross-section to FSI,
in which, $\sigma_{exp}\sim\sigma_{PWIA}$.  This kinematic window in which FSI are small can also be seen in the momentum distributions for $\theta_{nq}=35\pm5^{o}$ and $\theta_{nq}=45\pm5^{o}$, where data and theory agree well
within the PWIA kinematics. The experiment concluded that this kinematic window found at $35^{o}\leq \theta_{nq}\leq45^{o}$ gives for the first time a direct access to the high momentum components of the deuteron momentum distribution.



%Depending on the kinematics chosen, however, PWIA may be overshadowed by
%long-range processes such as final-state interactions (FSIs), meson-exchange currents (MECs) or isobar confgurations (ICs).




\input acknowledgement.tex   % input acknowledgement

\bibliography{prl}

\end{document}
%
% ****** End of file template.aps ******
