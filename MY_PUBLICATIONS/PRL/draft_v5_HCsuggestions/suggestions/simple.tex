

\title{E12-10-003 Hall C Collaborators Comments/Suggestions for PRL}
\author{
        Carlos Yero \\
                Department of Physics\\
        Florida International University\\
}
\date{\today}

\documentclass[12pt]{article}
\usepackage{xcolor}
\usepackage[margin=1in]{geometry}
\usepackage{hyperref}

\begin{document}
\maketitle


\section{Collaborator: Douglas Higinbotham}
Dear Carlos, \\
\newline
\noindent Congratulations on getting this together! I suspect you are going to need to tighten up the comments about the plots on page 4 to meet the PRL length limit, but in general it look good.
One important comment, the colors in the plots are impossible for someone who is red-green color blind.  (i.e. 8$\%$ of men and  $\sim$1$\%$ of women).
Try copying (using a color picker) the colors from this games made by our own Huey-Wen Lin et al.:\\
\newline
\url{https://play.google.com/store/apps/details?id=com.gellab.quantum3}  \\
\newline
While they don’t look red, green, blue to the color blind, they are three distinct shades which means everyone will be able to tell the three curves apart.
\newline
For the data points, please use two different shapes along with the color and that should be fine.
(e.g. circle and square)
\newline
Best Wishes,
Douglas

\textcolor{blue}{\section{Collaborator: Dave Mack}}
\textcolor{blue}{
        0. KUDOS - What a great draft! It was hard work to find even one typo. \\
        \newline
        1.  GETTING INTO PRL - To increase the chance of getting accepted by PRL, you should add a few points to make a stronger case that
      	  this measurement is a qualitative advance over the Hall A data.  (You made that case already well wrt the Hall B data.) On page
	  2 when you compare the C measurements vs A/B, while you do give the specific Pmiss values, I would emphasize that you're "almost 
	  doubling the maximum Pmiss  compared to the Hall A measurement".  Also, you should mention in this same section the Hall A Q2 of 3.5
	  since that question is begged. (Unfortunately, the higher Q2 of the new C data sounds more like a quantitative than a qualitative
	  improvement, a fact echoed in your title.)  You could also try playing the "new facility" card,  adding one sentence to point out
	  that these are the first results using the new SHMS in Hall C. \\
          \newline
        2. SYSTEMATIC UNCERTAINTIES - The uncertainty analysis on page 3 assumes that SIMC 1) reproduces all the data resolutions for both
      	 spectrometers, and 2) reproduces the acceptance for both spectrometers. That might have been ~true for Hall C in 2005. My impression
	 is that both are optimistic for the SHMS at this time.  This could add another few percent plus another few percent in quadrature to
	 your total systematic error. No shame in that for a commissioning experiment, especially one with even larger statistical errors. \\
         \newline
        3. MISSING ENERGY CUT QUESTIONS - I'd like some clarification on the missing energy (ME) cut for the larger Pmiss settings. This seems to
     	be an exclusivity cut to suppress inelasticities like n+pi0 for example. Carlos showed a lot of representative plots for the 80 MeV/c
	settings in his recent talk, and said on slide 29 the non-PID plots would be the same at higher Pmiss.  But the ME cut is not a PID cut
	per se, and would have to be adjusted for each range of Pmiss .  Can I see those ME plots for the higher Pmiss settings? 
}

\textcolor{violet}{\section{Collaborator: Garth Huberg}}
\textcolor{violet}{
Dear Carlos, Werner and Mark, \\
\newline
Thanks for the opportunity to comment on the draft manuscript.  Overall, the 
data look very nice. Congratulations on getting this done so quickly. \\
\newline
Some comments/questions I have:\\
\newline
1. Since this is the first SHMS+HMS physics paper, I wonder if some additional 
analysis details, such as maybe a coincidence time or missing mass spectrum 
would be helpful, provided there is room. \\
\newline
2. Similarly, more information should be given on the ``standard cuts'' on page 2, 
lower right, given this is a new spectrometer.  e.g. how wide is the missing 
energy cut around the 2.22 MeV peak?  How wide is the coincidence time cut?  How 
large is the region in which SHMS reconstruction optics are well known? \\
\newline
3. Most readers prefer information to be quantitative than qualitative.  Thus, I 
suggest ``Very High'' and ``Large'' in the title be replaced with numeric ranges. 
Similarly, in the second last paragraph "very high neutron recoil momentum" 
should be quantified.\\
\newline
4. why is the 75deg setting not mentioned in the Abstract?\\
\newline
5. I'm kind of confused by the definition of missing energy at bottom left of 
page 2, Em=omega-Tp-Tr.  Since Tr is not measured, I was expecting Em=omega-Tp. 
How is Tr determined in this case?  Do you need to assume knowledge of the 
missing momentum and an exclusive reaction when computing Tr?\\
\newline
6. Fig 2, there is wasted space at the tops of panels a,c.  Can the y-axis ranges 
be reduced for these?\\
\newline
There are quite a few places where the English could be improved.  They are 
listed at the bottom of this message.\\
\newline
Regards, Garth \\
\newline
\underline{GRAMMATICAL CORRECTIONS}\\
\newline
TITLE: D in D(e,e'p)n should not be capitalized.  Either use 2H (as in the rest 
of the paper), or d.\\
\newline
ABSTRACT:\\
Remove New in the first line.
At these kinematic settings COMMA final state interactions...
to recent theoretical calculations COMMA where...\\
\newline
Page 1, LHS:\\
at the sub-Fermi distance (i.e. Fermi must be capitalized)
At such small inter-nucleon distances COMMA the...
are also important in determining whether COMMA or to what extent COMMA the...
nucleon/meson degrees of freedom is still valid COMMA before having...\\
\newline
Page 2, LHS:\\
a strong angular dependence with neutron recoil angles COMMA peaking at...
liquid deuterium target (LD2) --> shouldn't the 2 be subscripted? \\
\newline
Page 2, RHS:\\
covered a range of 2.4<q<3.2 GeV/c COMMA which is...
neutron recoil momentum (pr) measured (BY or IN, not ON) this experiment.
At these forward angles and large momentum --> both angles and momentum need to 
be plural, so change momentum to momentA
data WERE also taken at kinematics close to the deutron...
for which the reconstruction optics ARE well known...
true coincidence events and not ACCIDENTAL COINCIDENCES (accidentals is jargon)
background (mostly pions) COMMA and a...
For 2H(e,e'p)n COMMA the corrections were \\
\newline
Page 3, LHS:\\
The good agreement gives us confidence (IN not ON)\\
\newline
Page 3, RHS:\\
by J.M. SPACE Laget including FSI...
kinematics WAS calculated and used in the bin centering...
averaged OVER overlapping bins in pr.
Eqn 1: needs a COMMA\\
\newline
Page 4, LHS:\\
Hall A experiment [11] at Q2=3.5 (i.e. remove A)
at recoil momenta pr<300 MeV/c COMMA the cross sections...
In addition COMMA for theta\_nq=35deg COMMA they are dominated...
For theta\_nq=35 and 45 (Figs. 2a,b) COMMA the data...
At larger recoil momenta COMMA where...\\
\newline
Page 4, RHS:\\
Overall COMMA it is interesting...\\
\newline
Page 5, LHS:\\
At higher missing momentA, ...
Ref 6: 4.0 GeV (not gev)\\
}
\textcolor{red}{\section{Collaborator: Florian Hauenstein}}
\textcolor{red}{
1. The HallA data is hard to distinguish in color code from the lines. Maybe try some orange/red color and also 'boxes' instead of 'circles'
   to indicate that it is a different data. Maybe also open boxes.\\
\newline
2. You describe the radiative and bin-centering corrections but not their corresponding systematic uncertainties.\\
\newline
3. Is pr=1.01GeV/c exactly the upper limit? If so it would be good to indicate it in the plots by using the horizontal size of the last bin
   or so. If not can't you write p\_r$\sim$1GeV/c?\\
\newline
4. I hope you will add a Table with all data points and their individual uncertainties in an extended information for the paper. The same holds
   for an equation with all corrections.\\
\newline
5. In a lot of cases in the text, there is a space missing between text and citation.\\
\newline
6. The explanation of the correction for efficiencies is weird given the numbers: tracking and live time is high but loss and boiling is low.
   I assume that the corrections are applied differently - some per division and some per multiplication? It is difficult to understand what
   is meant.\\
\newline
7. The recoil factor f\_Rec is not really explained. Maybe add a reference.\\
\newline
8.  It is not completely clear what one 'data set' is where you add later errors in quadrature.\\
\newline
9. Missing Reference to SIMC and HMS/SHMS spectrometers\\
\newline
10. When GEA is described, I would think more of $\theta_{nq} < 40$ than $\theta_{nq} \sim 40$ for low FSI. The same holds for $\theta_{nq} > 120$ instead
    of $\theta_{nq} \sim 120$\\
\newline
11. In the beginning when you cite references 1-4, you probably should add Barak's Nature paper (B. Schmookle, Nature 566, 354 (2019)) and
    Or's and Larry's Review (O. Hen et al., Rev.Mod Phys 89, 045002 (2017))\\
\newline
12. I would add one more sentence at the end, given an outlook on the next steps/experiments.\\
}
\end{document}
