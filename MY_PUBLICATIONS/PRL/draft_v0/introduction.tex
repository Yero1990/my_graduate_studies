Being the only two-nucleon bound system, the deuteron serves as a starting point to study the strong nuclear force at the subfermi level which is currently
not well understood. At such small internucleon distances the NN (nucleon-nucleon) potential is expected to exhibit a repulsive core in which the interacting
nucleon pair begins to overlap. The overlap is directly related to two-nucleon short range correlations (SRC) observed in $A>2$ nuclei \cite{PhysRevC.68.014313,PhysRevLett.96.082501,PhysRevLett.99.072501,Fomin_2017}.
Short-range studies of the deuteron are also important in determining whether or to what extent is the description of nuclei in terms of nucleon/meson degrees of freedom valid before
having to include explicit quark effects, which is an issue of fundamental importance in nuclear physics\cite{pr01-020}. As of the present time, there are only a few nuclear physics experiments for
which a transition between nucleonic to quark degrees of freedom been observed \cite{PhysRevLett.81.4576,PhysRevLett.87.102302,PhysRevC.66.042201}.
This Letter presents first results of $^{2}H(e,e'p)n$ in which kinematics were taken to the limit where a transition to non-nucleonic degrees of freedom is expected.\\
\indent The most direct way to study the short range structure of the deuteron wavefunction (or equivalently, its high momentum components) is via the exclusive deuteron
electro-disintegration reaction at very high neutron recoil (or missing) momenta and within the PWIA kinematics. In this approximation, the virtual photon couples to
the proton which is ejected from the nucleus without further interaction with the recoiling neutron, which carries a momentum equal in magnitude but opposite in direction
to the initial state proton, $\vec{p}_{r} = -\vec{p}_{i,p}$. This gives direct access to the deuteron momentum distributions since the scattered neutron momentum remains unchanged from its initial state. \\
\indent In reality, the ejected particles undergo subsequent interactions resulting in re-scattering between the proton and neutron (FSIs). Another possibility is that the
photon may couple to the virtual meson being exchanged between the nucleons (MECs), or the photon may excite either nucleon in the deuteron into a resonance state (ICs) which
decays back into the ground state nucleon causing futher re-scattering between the proton and neutron. Both MECs and ICs in addition to FSIs can significantly alter the recoiling neutron
momentum thereby obscuring any possibility of directly accessing the deuteron momentum distributions. \\
%(1) The above-mentioned long-range processes alter the final neutron
%momentum making the deuteron momentum distributions difficult to access.\\
\indent Previous deuteron electro-disintegration experiments performed at Jefferson Lab (JLab) have helped constrain and quantify the contributions from FSIs, MECs and ICs on
the $^{2}H(e,e'p)n$ cross-section and determine the kinematics at which they are either suppressed (MECs and ICs) or under control (FSIs). The first of these was performed in Hall A \cite{PhysRevLett.89.062301}
at a relatively low momentum transfer of $Q^{2}=0.665$ GeV$^{2}$ and neutron recoil momenta
up to $p_{r} = $ 550 MeV/c where it was shown that for $p_{r}>$ 300 MeV/c, FSIs, MECs and ICs played a significant role and had to be included in Arenh\"{o}vel's calculations \cite{PhysRevC.43.1022, PhysRevC.46.455, PhysRevC.52.1232, PhysRevC.55.2214}
for a satisfactory agreement between theory and data. \\
%(2) the inclusion of FSI, MEC and IC was necessary in Arenhovel's calculations for a satisfactory
%agreement between the theory and data. \\
\indent The next experiment was performed in Hall B \cite{PhysRevLett.98.262502} using the CEBAF Large Acceptance Spectrometer (CLAS) which measured a wide variety of kinematic settings
giving an overview of the $^{2}H(e,e'p)n$ reaction kinematics. This was the first experiment to probe
the deuteron at high momentum transfers ( 1.75 $\leq Q^{2}\leq$ 5.5 GeV$^{2}$) and presented angular distributions of cross-sections that confirmed the onset of
the General Eikonal Approximation (GEA)\cite{sargsian_2001,PhysRevC.56.1124}, which predicts a strong angular dependence of FSI with neutron recoil angles with a peak at $\theta_{nq} \sim 70^{o}$.
The cross sections versus neutron recoil momenta up to $p_{r}\sim$2 GeV/c were also presented, however, statistical limitations made it impossible to select kinematical bins in $\theta_{nq}$ where
FSIs were small to extract momentum distributions.\\
% (3) The cross-sections versus neutron recoil momenta up to 2 GeV/c were also presented with integrated neutron recoil angles in the range $20^{o}< \theta_{nq}<160^{o}$
%to gain better statistical precision. As a result, it was not possible to choose kinematical regions binned in $\theta_{nq}$ in which FSI were minimal to extract the momentum distributions. \\
\indent Finally, a third $^{2}H(e,e'p)n$ experiment was performed in Hall A \cite{PhysRevLett.107.262501} at published $Q^{2} = 3.5\pm0.25$ GeV$^{2}$ and recoil momenta up to 550 MeV/c. The angular distributions of
the cross-section ratio ($R = \sigma_{exp}/\sigma_{PWIA}$) presented verified the strong anisotropy of FSIs with recoil angle $\theta_{nq}$ also observed in Hall B\cite{PhysRevLett.98.262502}. Most importantly, for recoil
neutron momentum bins, $p_{r}=0.4\pm0.02$ and $0.5\pm0.02$ GeV/c, the ratio was found to be $R\sim1$ for $35^{o}\leq \theta_{nq}\leq45^{o}$ indicating a reduced sensitivity of the experimental
cross-section to FSIs. This kinematic window allowed for the first time the extraction of momentum distributions for neutron recoil momenta up to $p_{r}\sim$550 MeV/c. \\
% (4) This kinematic window in which FSI are small can also be seen in the momentum distributions for
%$\theta_{nq}=35\pm5^{o}$ and $45\pm5^{o}$, where data and theory agree well within the PWIA kinematics.
\indent The experiment presented on this Letter takes advantage of the kinematic window previosuly found in Hall A\cite{PhysRevLett.107.262501} and extends the $^{2}H(e,e'p)n$ cross section measruements
to $Q^{2}=4.5\pm0.5$ GeV$^{2}$ and neutron recoil momenta up to 1.18 GeV/c. At these kinematics, MECs and ICs are suppressed and FSIs are under control for neutron recoil angles between 35 and 45 degrees
giving access to unprecedented high momentum components of the deuteron wavefunction. \\
