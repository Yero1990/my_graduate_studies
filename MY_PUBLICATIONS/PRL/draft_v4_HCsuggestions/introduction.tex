The deuteron is the only bound two-nucleon  system and serves as a starting point to study the strong nuclear force at the sub-Fermi distance scale, a region which is currently
not well understood. At such small inter-nucleon distances, the nucleon-nucleon (NN) interaction is expected to become repulsive and the interacting
nucleons begin to 
This short distance region is directly related to two-nucleon short range correlations (SRC) observed in $A>2$ nuclei \cite{PhysRevC.68.014313,PhysRevLett.96.082501,PhysRevLett.99.072501,Fomin_2017,Barack_2019,RevModPhys.89.045002}.
Short-range studies of the deuteron are also important in determining whether, or to what extent, the description of nuclei in terms of nucleon/meson degrees of freedom is still valid, before
having to include explicit quark degree of freedoms, an issue of fundamental importance in nuclear physics \cite{sargsian_2015}. As of the present time, there are only a few nuclear physics experiments for
which a transition between nucleonic to quark degrees of freedom has been observed \cite{PhysRevLett.81.4576,PhysRevLett.87.102302,PhysRevC.66.042201}. \\
\indent The most direct way to study the short range structure of the deuteron wave function (or equivalently, its high momentum components) is via the exclusive deuteron
electro-disintegration reaction at $p_{r}>300$ MeV/c neutron recoil (or ``missing'') momenta. Within the PWIA, the virtual photon couples to
the bound proton which is subsequently ejected from the nucleus without further interaction with the recoiling system (neutron). The neutron carries a momentum equal in magnitude but opposite in direction
to the initial state proton, $\vec{p}_{\mathrm{r}} = -\vec{p}_{\mathrm{i},p}$, thus providing information on the momentum of the bound nucleon and its momentum distribution. \\
\indent In reality, the ejected nucleon undergoes FSI corresponding to subsequent interactions with the recoiling system. Another possibility is that the
photon couples to the virtual meson being exchanged between nucleons (MEC), or that the photon excites a bound nucleon into a resonance state which subsequently
decays back into its ground state (IC).  FSI, MEC and IC can significantly alter the recoiling neutron
momentum thereby obscuring the original momentum of the bound nucleon and reducing the possibility of directly probing the deuteron momentum distribution. \\
\indent Theoretically, MEC and IC are expected to be suppressed at $Q^{2}>1$ (GeV/c)$^{2})$ and Bjorken $x_{\mathrm{Bj}}\equiv Q^{2}/2M_{p}\omega>1$, where $M_{p}$ and $\omega$ are the proton mass and photon energy transfer, respectively.
The suppression of MEC can be understood from the fact that the estimated MEC scattering amplitude is proportional to  $(1 + Q^{2}/m^{2}_{\mathrm{meson}})^{-2}(1+Q^{2}/\Lambda^{2})^{-2}$, where $m_{\mathrm{meson}}\approx0.71$ GeV/c$^{2}$ and
$\Lambda^{2}\sim 0.8-1 $ (GeV/c)$^{2}$ \cite{Sargsian_2001}. IC can be suppressed kinematically by selecting $x_{\mathrm{Bj}}>1$, where one probes the lower energy part of the deuteron quasi-elastic peak which is maximally far away from the inelastic resonance
electro-production threshold.\\
\indent Previous deuteron electro-disintegration experiments performed at lower $Q^{2}$ ($Q^{2}<1$ (GeV/c)$^{2}$)(See Section 5 of Ref. \cite{sargsian_2015}) have helped quantify the contributions
from FSI, MEC and IC on the $^{2}\mathrm{H}(e,e'p)n$ cross-section and to determine the kinematics at which they are either suppressed (MEC and IC) or under control (FSI).  \\
\indent At large $Q^{2}$, FSI are described by the General Eikonal Approximation (GEA) \cite{Sargsian_2001,PhysRevC.56.1124,sargsian_2015} which predicts a strong dependence of FSI on neutron recoil angles $\theta_{nq}$.
GEA predicts FSI to be maximal for $\theta_{nq}\sim70^{\circ}$. This strong angular dependence has been found to lead to the cancellation of FSI at neutron recoil angles around $\theta_{nq}\sim40^{\circ}$ and $\theta_{nq}\sim120^{\circ}$. Since at $\theta_{nq}\sim120^{\circ}$ IC are not negligible, $x_{\mathrm{Bj}}>1$ (or equivalently $\theta_{nq}\sim40^{\circ}$) is the preferred choice to suppress IC as well as FSI. \\
%---------D(e,e'p) Experiments at Q2 < 1-------- 
\indent The first $^{2}\mathrm{H}(e,e'p)n$ experiments at high $Q^{2}$ ($>1$ (GeV/c)$^{2}$) were carried out at Jefferson Lab (JLab) in Halls A \cite{PhysRevLett.107.262501} and B \cite{PhysRevLett.98.262502}. Both
experiments determined that the cross-sections for fixed missing momenta indeed exhibited a strong angular dependence with neutron recoil angles, peaking
at $\theta_{nq} \sim 70^{\circ}$ in agreement with GEA \cite{Sargsian_2001,PhysRevC.56.1124} calculations. In Hall B, the CEBAF Large Acceptance Spectrometer (CLAS) measured angular
distributions for  a range of $Q^2$ values as well as momentum distributions. However, statistical limitations made it necessary to integrate over a wide angular range to determine momentum distributions
which are therefore dominated by  FSI, MEC and IC for missing momenta above $\sim 300$ MeV/c. \\
\indent In Hall A, the pair of high resolution spectrometers (HRS) made it possible to measure the missing momentum dependence of the cross section for fixed neutron recoil angles ($\theta_{nq}$) reaching missing momenta up to $p_{\mathrm{r}}=550$ MeV/c at $Q^{2}=3.5\pm0.25$ (GeV/c)$^{2}$. For the first time very different momentum distributions were found for $\theta_{nq}=35\pm5^{\circ}$
and $45\pm5^{\circ}$ compared to  $\theta_{nq}=75\pm5^{\circ}$. Theoretical models attributed this difference  to the suppression of FSI at the smaller angles ($\theta_{nq}=35, 45^{\circ}$) compared to FSI
dominance at $\theta_{nq}=75^{\circ}$ \cite{PhysRevLett.107.262501}. \\
\indent The experiment presented in this Letter takes advantage of the kinematic window previously found in the Hall A experiment and extends the $^{2}\mathrm{H}(e,e'p)n$ cross section measurements
to $Q^{2}=4.5\pm0.5$ (GeV/c)$^{2}$ and neutron recoil momenta up to $p_{r}\sim 1$ GeV/c which is almost double of the maximum recoil momentum measured in Hall A \cite{PhysRevLett.107.262501}.
Measurements at such large $Q^{2}$ and high missing momenta required scattered electrons to be detected at about 8.5 GeV/c which was only made possible with the newly commissioned Hall C Super High Momentum Spectrometer (SHMS).
At the selected kinematic settings with neutron recoil angles between $35^{\circ}$ and $45^{\circ}$, MEC and IC are suppressed and FSI are under control giving access to high momentum components of the deuteron wave function.


