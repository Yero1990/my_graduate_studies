Being the most simple $np$ bound state, the deuteron serves as a starting point to study the strong nuclear force at the subfermi level which is currently
not well understood. At such small internucleon distances the NN (nucleon-nucleon) potential is expected to exhibit a repulsive core in which the interacting
nucleon pair begins to overlap. The overlap is directly related to two-nucleon short range correlations (SRC) observed in $A>2$ nuclei \cite{PhysRevC.68.014313,PhysRevLett.96.082501,PhysRevLett.99.072501,Fomin_2017}. Short-distance studies
of the deuteron are also important in determining whether or to what extent the description of nuclei in terms of nucleon/meson degrees of freedom must be supplemented
by the inclusion of explicit quark effects, which is an issue of fundamental importance in nuclear physics\cite{pr01-020}. To date, there are only a few nuclear experiments for
which a transition between nucleonic to quark degrees of freedom can be observed \cite{PhysRevLett.81.4576,PhysRevLett.87.102302,PhysRevC.66.042201}. This Letter presents
first results of $^{2}H(e,e'p)n$ in which kinematics were taken to the limit where a transition to non-nucleoninc degrees of freedom is expected.\\
\indent The most direct way to study the short range structure of the deuteron wavefunction (or equivalently, its high momentum components) is via the exclusive deuteron
electro-disintegration reaction at very high neutron recoil (or missing) momenta and within the PWIA kinematics. In this approximation, the virtual photon couples to
the proton which is ejected from the nucleus without further interaction with the recoiling neutron, which carries a momentum equal in magnitude but opposite in direction
to the initial state proton, $\vec{p}_{r} = -\vec{p}_{i,p}$. This gives direct access to the deuteron momentum distributions since the scattered neutron momentum remains unchanged. \\
\indent In reality, the ejected particles undergo subsequent interactions resulting in re-scattering of the proton and neutron (FSIs). Another possibility is that the
photon may couple to the virtual meson being exchanged between the nucleons (MECs), or the photon may excite either nucleon in the deuteron into a resonance state (ICs) which
decays back into the ground state nucleon causing futher re-scattering between the proton and neutron. The above-mentioned long-range processes alter the final neutron
momentum making the deuteron momentum distributions difficult to access.\\
\indent Previous deuteron electro-disintegration experiments performed at Jefferson Lab (JLab) have helped dis-entangle and quantify the contributions from FSI, MEC and IC on
the $^{2}H(e,e'p)n$ cross-section and determine the kinematics at which they are either suppressed (MECs and ICs) or under control (FSIs). The first of these was performed in Hall A  \cite{PhysRevLett.89.062301} at a relatively low momentum transfer of $Q^{2}=0.665$ (GeV/c)$^{2}$ and neutron recoil momenta
up to $p_{r} = $ 550 MeV/c where it was shown that for $p_{r}>$ 300 MeV/c, the inclusion of FSI, MEC and IC was necessary in Arenhovel's calculations for a satisfactory
agreement between the theory and data. \\
\indent The next experiment was performed in Hall B \cite{PhysRevLett.98.262502} using the CEBAF Large Acceptance Spectrometer (CLAS) which took advantage of its large detector acceptance to
simultaneously measure a wide variety of kinematic settings giving an overview of the $^{2}H(e,e'p)n$ reaction kinematics. This was the first experiment to probe
the deuteron at high momentum transfers ( 1.75 $\leq Q^{2}\leq$ 5.5 (GeV/c)$^{2}$) and presented angular distributions of cross-sections that confirmed the onset of
the General Eikonal Approximation (GEA), predicting a strong angular dependence of FSI with neutron recoil angles with FSI peaking at $\theta_{nq} \sim 70^{o}$.
The cross-sections versus neutron recoil momenta up to 2 GeV/c were also presented with integrated neutron recoil angles in the range $20^{o}< \theta_{nq}<160^{o}$
to gain better statistical precision. As a result, it was not possible to choose kinematical regions binned in $\theta_{nq}$ in which FSI were minimal to extract the momentum distributions. \\
\indent Finally, a third $^{2}H(e,e'p)n$ experiment was performed in Hall A \cite{PhysRevLett.107.262501} at $Q^{2} = 0.8, 2.5, 3.5$ (GeV/c)$^{2}$ and recoil momenta up to 550 MeV/c at kinematics
which allowed the extraction of angular and momentum distributions for significantly smaller kinematical bins than in Hall B/CLAS. The angular distributions were presented
as the cross-section ratio, $R = \sigma_{exp}/\sigma_{PWIA}$ versus $\theta_{nq}$, and verified the strong anisotropy of FSI with recoil angle previously observed in Hall B. Most importantly, for recoil
neutron momentum bins, $p_{r}=0.4\pm0.02$ and $0.5\pm0.02$ GeV/c, the ratio $R\sim1$ for $35^{o}\leq \theta_{nq}\leq45^{o}$ indicating a reduced sensitivity of the experimental
cross-section to FSI, in which, $\sigma_{exp}\sim\sigma_{PWIA}$.  This kinematic window in which FSI are small can also be seen in the momentum distributions for
$\theta_{nq}=35\pm5^{o}$ and $45\pm5^{o}$, where data and theory agree well within the PWIA kinematics. The experiment concluded that the kinematic window found
at $35^{o}\leq \theta_{nq}\leq45^{o}$ gives for the first time a direct access to the high momentum components of the deuteron momentum distribution. \\
\indent This experiment takes advantage of the kinematic window found previosuly in Hall A and extends the $^{2}H(e,e'p)n$ cross section measruements to $Q^{2}=4.5$ (GeV/c)$^{2}$
and neutron recoil momenta up to 1.18 GeV/c. In this configuration, MECs and ICs are suppressed and FSIs are under control for recoil angles between 35 and 45 degrees
giving access to unprecedented high momentum components of the deuteron wavefunction. \\
