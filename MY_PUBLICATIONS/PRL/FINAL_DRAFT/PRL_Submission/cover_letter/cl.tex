%basic cover letter template
\documentclass{letter}
\usepackage{amssymb,amsmath}
\usepackage{graphicx}
\usepackage{hyperref}
\oddsidemargin=.2in
\evensidemargin=.2in
\textwidth=5.9in
\topmargin=-.5in
\textheight=9in

%\address{Mathematics Department\\University of Illinois\\
%1409 W. Green St\\Urbana, Illinois 61801}

\newcommand {\qed}{\mbox{$\Box$}}
\renewcommand {\iff}{\Longleftrightarrow}
\newcommand {\R}{\mathbb{R}}
\newcommand {\N}{\mathbb{N}}
\newcommand {\Q}{\mathbb{Q}}
\newcommand {\Z}{\mathbb{Z}}

\newcommand {\sub}{\mbox{SB}}


\begin{document}
\begin{letter}{
    Dr. Michael Thoennessen\\
    Editor-in-Chief, APS Editorial Office \\
    \textit{Physical Review Letters} (PRL)\\
    }


\opening{Dear Dr. Thoennessen,\\}

I hereby submit a manuscript entitled ``Probing the Deuteron at Very Large Internal Momenta'' to be considered for publication by \textit{Physical Review Letters}.
There are eight files in all: the main manuscript file (ms.tex), 2 figure files (.pdf), a bibliography file (.bbl), an author list file (.tex), and three supplementary (.txt)
files with the numerical results of the cross sections evaluated at the relevant kinematics described in the manuscript.\\

We report cross section measurements of the $^{2}$H$(e,e'p)n$ at neutron recoil momenta up to $p_{\mathrm{r}}\sim$ 1.0 GeV/c which is almost
double the maximum recoil momentum measured in a previous deuteron electro-disintegration experiment carried out in Hall A at Jefferson Lab. 
The Hall A experiment concluded that at neutron recoil angles $35^{\circ}\leq \theta_{nq}\leq 45^{\circ}$, final state interactions (FSI) are largely
reduced and the plane wave impulse approximation (PWIA) provides the dominant contribution to the cross section. This experiment focuses
on the kinematic window (where FSI are small) previously found in Hall A and extends the missing momentum coverage from $p_{\mathrm{r}}\sim0.5 - 1.0$ GeV/c, giving
direct access to the high momentum component of the deuteron wave-function. This is a crucial step forward in the understanding the nucleon-nucleon ($NN$) interaction at
the sub-Fermi distance scale which is currently poorly understood. We believe these findings will be of general interest to the readers of your journal.\\

We declare that this manuscript is original, has not been published before in a peer-reviewed journal and is not currently being considered for publication elsewhere.\\
\\
\closing{Sincerely,
%\includegraphics[width=6cm]{signature.pdf}
}
%I made a pdf file of my signature using the scanner, so that all the cover letters I sent
%electronically were "signed."  
Carlos Yero\\
Hall C Postdoctoral Research Fellow\\
Thomas Jefferson National Accelerator Facility\\
12000 Jefferson Avenue \\
Newport News, VA 23602\\
e-mail: cyero@jlab.org\\
phone: (786) 387-8667 

\end{letter}

\end{document}

