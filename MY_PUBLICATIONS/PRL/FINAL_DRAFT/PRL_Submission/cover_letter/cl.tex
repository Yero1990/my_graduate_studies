%basic cover letter template
\documentclass{letter}
\usepackage{amssymb,amsmath}
\usepackage{graphicx}
\usepackage{hyperref}
\oddsidemargin=.2in
\evensidemargin=.2in
\textwidth=5.9in
\topmargin=-.5in
\textheight=9in

%\address{Mathematics Department\\University of Illinois\\
%1409 W. Green St\\Urbana, Illinois 61801}

\newcommand {\qed}{\mbox{$\Box$}}
\renewcommand {\iff}{\Longleftrightarrow}
\newcommand {\R}{\mathbb{R}}
\newcommand {\N}{\mathbb{N}}
\newcommand {\Q}{\mathbb{Q}}
\newcommand {\Z}{\mathbb{Z}}

\newcommand {\sub}{\mbox{SB}}


\begin{document}
\begin{letter}{Carlos Yero\\
Thomas Jefferson National Accelerator Facility\\
12000 Jefferson Avenue \\
Newport News, VA 23602\\
(786) 387-8667 \\
cyero@jlab.org
  }


\opening{Dear Editors,\\}

I hereby submit a manuscript entitled ``Probing the Deuteron at Very Large Internal Momenta'' to be considered
for publication by \textit{Physical Review Letters}. There are eight files in all: the main manuscript file (prl.tex), 2 figure files (.pdf),
a biblipgraphy file (.bbl), an author list file (.tex), and three supplementary (.txt) files with the numerical results of the
cross sections evaluated at the relevant kinematics described in the manuscript.\\

We report cross-section measurements of the $^{2}$H$(e,e'p)n$ at neutron recoil momenta up to $p_{\mathrm{r}}\sim$ 1.0 GeV/c which is almost
double the maximum recoil momentum measured in a previous deuteron electro-disintegration experiment carried out in Hall A at Jefferson Lab. 
That experiment concluded that at neutron recoil angles $35^{\circ}\leq \theta_{nq}\leq 45^{\circ}$, final state interactions (FSI) are largely
reduced and the plane wave impulse approximation (PWIA) provided the dominant contribution to the cross section. This experiment focused
on the kinematic window (where FSI are small) found by the Hall A experiment and extended the missing momentum coverage from $p_{\mathrm{r}}\sim0.5 - 1.0$ GeV/c, giving
access to the high momentum component of the deuteron wave-function, or equivalently, the nucleon-nucleon ($NN$) interaction at the sub-Fermi
distance scale which is currently poorly understood.\\

This Letter summarizes the experimental results of cross-section measurements that extended the neutron recoil momentum measured up to $p_{\mathrm{r}}\sim$ 1.0 GeV/c giving a direct access
to the shot-range structure of the deuteron wave-function which is a crucial step forward in the understanding the $NN$ interaction at very small ($\leq$1 fm) distances. We believe these
findings will be of general interest to the readers of your journal.\\

We declare that this manuscript is original, has not been published before and is not currently being considered for publication elsewhere.\\
\\
\closing{Sincerely,
%\includegraphics[width=6cm]{signature.pdf}
}
%I made a pdf file of my signature using the scanner, so that all the cover letters I sent
%electronically were "signed."  

Carlos Yero\\
Hall C Postdoctoral Fellow\\
\end{letter}

\end{document}

