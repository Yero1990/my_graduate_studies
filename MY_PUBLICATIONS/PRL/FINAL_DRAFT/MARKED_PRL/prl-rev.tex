%% ****** Start of file template.aps ****** %
%%
%%
%%   This file is part of the APS files in the REVTeX 4 distribution.
%%   Version 4.0 of REVTeX, August 2001
%%
%%
%%   Copyright (c) 2001 The American Physical Society.
%%
%%   See the REVTeX 4 README file for restrictions and more information.
%%
%
% This is a template for producing manuscripts for use with REVTEX 4.0
% Copy this file to another name and then work on that file.
% That way, you always have this original template file to use.
%
% Group addresses by affiliation; use superscriptaddress for long
% author lists, or if there are many overlapping affiliations.
% For Phys. Rev. appearance, change preprint to twocolumn.
% Choose pra, prb, prc, prd, pre, prl, prstab, or rmp for journal
%  Add 'draft' option to mark overfull boxes with black boxes
%  Add 'showpacs' option to make PACS codes appear
\documentclass[aps,prl,twocolumn,showpacs,superscriptaddress,groupedaddress]{revtex4-2}  % for review and submission
%\documentclass[aps,preprint,showpacs,superscriptaddress,groupedaddress]{revtex4}  % for double-spaced preprint
\usepackage{graphicx}  % needed for figures
\usepackage{ragged2e}
\usepackage{dcolumn}   % needed for some tables
\usepackage{bm}        % for math
\usepackage{amssymb}   % for math
\usepackage[hidelinks]{hyperref}
\usepackage{fancyhdr}
\usepackage{xcolor}
\usepackage{float}
\usepackage{lipsum}
\hypersetup{
    colorlinks,
    linkcolor={blue!80!black},
    citecolor={blue!80!black},
    urlcolor={blue!80!black}
}
% avoids incorrect hyphenation, added Nov/08 by SSR
\hyphenation{ALPGEN}
\hyphenation{EVTGEN}
\hyphenation{PYTHIA}
%

\begin{document}

% The following information is for internal review, please remove them for submission
\widetext
\leftline{Version 06 as of \today}
\leftline{Primary authors: Carlos Yero, Werner Boeglin, Mark Jones}
\leftline{To be submitted to PRL}
\leftline{Comment to {\tt cyero002@fiu.edu, \tt boeglinw@fiu.edu, \tt jones@jlab.org} by July 31, 2020}
\centerline{INTERNAL DOCUMENT -- NOT FOR PUBLIC DISTRIBUTION}

% the following line is for submission, including submission to the arXiv!!
%\hspace{5.2in} \mbox{Fermilab-Pub-04/xxx-E}

%\title{First Measurements of the D(e,e'p)n Cross Section at \\ Very High Recoil Momenta and Large Q$^{2}$}
%\title{$^{2}$H$(e,e'p)n$ Cross Section Measurements for Neutron Recoil Momenta up to $p_{r}\sim1.01$ GeV/c at $Q^{2}=4.5\pm0.5$ (GeV/c)$^{2}$}
%\title{Deuteron Electro-Disintegration up to $\sim$ 1 GeV/c Missing Momentum}
\title{Probing the Deuteron at Very Large Internal Momenta}

\input authorlist.tex

\date{\today}


\begin{abstract}
  $^{2}\mathrm{H}(e,e'p)n$ cross sections have been measured at 4-momentum transfers of $Q^{2} = 4.5 \pm 0.5$ (GeV/c)$^{2}$
  over a range of neutron recoil momenta, $p_{\mathrm{r}}$, with $p_{\mathrm{r}}$ reaching up to $\sim1.0$ GeV/c. The data were
  obtained at fixed neutron recoil angles $\theta_{nq} = 35^\circ$, $45^\circ$ and $75^{\circ}$  with respect to the 3-momentum
  transfer $\vec q$. The new data agree well with previous data which reached $p_{\mathrm{r}}\sim500$ MeV/c. At $\theta_{nq} = 35^\circ$
  and $45^\circ$, final state interactions (FSI), meson exchange currents (MEC) and isobar configurations (IC) are suppressed and
  the plane wave impulse approximation (PWIA) provides the dominant cross section contribution. The new data are compared to recent
  theoretical calculations, where we observe a significant discrepancy for missing momenta $p_{\mathrm{r}}>700$ MeV/c. 
\end{abstract}

\pacs{}
\maketitle
%\thispagestyle{plain} 

%\section{\label{sec:level1}First-level heading}
% sections are not used for PRL papers

%--------------MOTIVATION / introduction (from my proposal)--------------

\input introduction.tex
\input body.tex
%\input acknowledgement.tex    % input acknowledgement
\bibliography{prl}
%.\\\\\\\\\\\\\\\\\\\\\\\\\\\\\\\\\\\\\\\\\\\\\\\\\\\\\\\\\\\\\\\\\\\\\\\\\\\\\\\\\\\\\\\\\\\\\\\\\\\\\\\\\\\\\\\\\\\\\\\\\\\\\\\\\\\\\\\\\\\\\\\\\\\\\\\\\\\
\end{document}
%
% ****** End of file template.aps ******
