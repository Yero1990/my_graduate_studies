\documentclass[a4paper]{article}
%\usepackage{simplemargins}

%\usepackage[square]{natbib}
%\usepackage{amsmath}
%\usepackage{amsfonts}
%\usepackage{amssymb}
%\usepackage{graphicx}
%\usepackage[T1]{fontenc}
%\usepackage[utf8]{inputenc}
\usepackage[english]{babel}
%\usepackage[urw-garamond]{mathdesign}
\usepackage[T1]{fontenc}
\usepackage[utf8]{inputenc}

\usepackage{tgbonum}

\begin{document}
\pagenumbering{gobble}

\Large
 \begin{center}
Deuteron Electro-Disintegration at Very High Missing Momenta\\ 

\hspace{10pt}

% Author names and affiliations
\large
Carlos Yero \\

\hspace{10pt}

April 07, 2019 \\

\hspace{10pt}

\small  
Florida International University\\

\end{center}

\hspace{10pt}

\normalsize

\noindent The Deuteron Electro-Disintegration experiment (E12-10-003) at Jefferson Lab, Hall C is discussed.  The experiment
consists of using an electron beam to break up the Deuteron into a proton and neutron. The scattered electron is detected in
coincidence with the knocked out proton and the missing neutron is reconstructed from conservation laws. The newly commissioned Super High
Momentum Spectrometer (SHMS) and existing High Momentum Spectrometer (HMS) are used to detect the electron and proton, respectively. The
D(e,e'p)n reaction investigates an unexplored kinematic region in which Final State Interactions (FSI) between the proton and neutron are small. In this
regime, the neutron missing momentum is approximately equal to its internal momentum in the Deuteron, enabling the extraction of momentum distributions
from the measured cross-sections. A general theoretical background of the D(e,e'p) reaction and very preliminary results are presented.



\end{document}
